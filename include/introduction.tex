\chapter{Introduction}
Binary rewriting is a method to update binaries with new
features, e.g., new ISA instructions, platform optimizations,
security checks for untrusted binaries, improved reliability.
%

Some of these features are easier to add when working at the binary-level
where information needed for the transformation is available.
Passing this information up to the appropriate statements at the
source-level is often not possible.
%
Furthermore, modifications performed at the source-level might be discarded
since the compiler only considers semantic correctness and
possibly improved execution time.
%

A binary rewriting framework can aid in development of binary
transformations by providing basic functions that can be
used when creating transformations and optimizations that could
be used to improve performance.
%

\section{Background}
\todo[inline]{Background on binary rewriting. Areas of use. More stuff}


\section{Goals}
The aim of the thesis is to create a framework for binary rewriting for MIPS
binaries in the open-source ROSE compiler infrastructure~\cite{rose}.
To show the benefits of a framework like this we will use it to implement
reliability transformations, which will be used to prove its usefulness.
%

The framework will help with implementing binary rewriting for
any purpose by providing basic functions that can be used when creating
transformations. A limited number of high-level concepts such as variables
will be available in the framework to remove some of the limitations of working
at the binary-level, e.g., finite amount of registers.
%

%limited amount of opimization mentioned her as well.
Furthermore, the framework will also provide a limited amount of optimizations
such as instruction scheduling and register allocation schemes that can
be used on transformed binaries to improve performance. These optimizations
provided also have to guarantee to not degrade applied transformations.


\section{Related Work}
%
There already exists several binary rewriting frameworks that are similar
to the framework we are proposing. However, these frameworks have
implementations that impedes the creation of transformed binaries
which are also well optimized. Two similar frameworks to our is SWIFT and
SecondWrite.
%

SWIFT uses compiler-based transformations to increase reliability~\cite{swift}.
A transformation it is capable of performing instruction duplication. This
creates two independant executions of the original program that can be
compared for consistency.
%
The transformations SWIFT have are based on templates and does not
have any optimizations that could be used to improve performance.
%Our framework will not apply transformations through templates and
%will as previously mentioned provide some global optimizations.
%

SecondWrite is a framework that is targeting x86~\cite{secondwrite}.
A binary that SecondWrite is going to transform
is translated into a intermediate representation(IR) with a higher
level of abstraction than instruction-level, the transformations
and optimizations are performed on this IR.
%
A drawback of this translation is that information is lost
which could be critical in determining if an optimization
should be used.
%
%Our framework will not use an IR and instead work
%at the binary-level. This is to preserve as much information as
%possible that can be used when applying transformations
%and optimizations.
%


\section{Method of Approach}

%Create basic framework components.
Initially the basic components of the framework will be
constructed, some of these will be used by applications
using the framework. Together with the basics of the framework
a naive transformation will be implemented, it will produce
functioning transformed binaries but with no regards for performance.
%

%implement optimizations
After the foundation of the framework and the naive transformation
is implemented then optimizations will be added to the framework.
The optimizations will provide users of the framework a
transformation applies optimizations to the transformation.

%Comparison between naive and optimized
The framework will be evaluated in terms of performance by
comparing transformed binaries with optimizations to
unoptimized transformed binaries. The Gem5 simululator will
be used for the evaluation of the binaries.

\section{Limitations}
Limitations on the project have been set since binary rewriting 
is a wide subject and there is a limited amount of time available
for the project.

\subsection{Partial Transformations}
%transformation of specific parts of a program. (kernel)
Instead of applying the transformations to an entire program
it will instead be used on selected functions.

\subsection{MIPS Instruction Set}
%mips only, release 1
%not floating point.
MIPS has been selected since it is a simple architecture compared
to x86 for example. Furthermore the project will limit itself
to the instruction set of MIPS Release I and will also not
include floating point instructions in any way.  

\subsection{Limited Amount of Optimizations}
%two optimizations
The number of transformations that will be provided has been
limited to two, one register allocation and one instruction
scheduling optimization.

\subsection{Fault-Tolerant Transformations}
As mentioned there are many applications for binary rewriting
but for this project fault-tolerant transformations have
been selected to be used. 


