\chapter{Framework Implementation}
%chaptermark is used to get a shorter name in the header.
\chaptermark{Implementation}

\begin{itemize}
    \item introduction to chapter
    \item ---------------------------------------
    \item common/general components
    \item rose
    \item addresses
    \item stack tracking
    \item activation records
    \item symbolic registers
    \item ---------------------------------------
    \item user transformation
    \item ---------------------------------------
    \item naive implementation
    \item ---------------------------------------
    \item optimized implementation
\end{itemize}


The implementation of the framework can be broken down to several components.
Some of these are common and needed in all transformations regardless
of if naive or optimized transformation is being used.

\section{Common Components}
The common components are such that they are needed to be able
to complete a transformation. All these components provide a
specific function needed during a transformation.

\subsection{ROSEs Frontend \& Backend}
%Since Rose is going to be used this will probably be quite short.

\subsection{Symbolic Registers}
%Describe the symbolic register use.



\subsection{Instruction Addresses}
%Why the address scope needs to be checked.
%Relates to the size of the binary.
When instructions are inserted or removed the addresses of the
subsequent instructions have to be changed to guarantee that
addressing is still consistent in the binary. The act of insertion
and removal also affect the size of the binary and the segments
within the binary, these changes need to be monitored so that
the sizes and ranges of segments can be corrected as well.


\subsection{Branch Tracking}
%Why do branch addresses needs to be tracked.
\todo[inline]{Merge this with instruction addresses?}
When instructions are removed or inserted in the binary the address
space of the binary is affected, branching offsets become larger
or smaller. The tracking branch offsets is needed to be able to correct
branch instructions after transformations have been applied.
If it is not corrected there is a significant risk that
the execution of the transformed binary does not correspond to that of the
original binary.

\subsection{Activation Records}
%Why the activation records needs to be controlled.
If there is a need to push more data to the stack due to transformations, e.g. 
more register than available are needed forcing values to be pushed to the stack. 
To avoid corruption on the stack the activation records which allocates space
on the stack needs to be monitored and modified correctly if more space is needed.


\section{Naive Transformation}
%what is special to the naive optimization
%The only concern the naive transformation method has is that it should only produce
%functional binaries with no concern for performance. Therefore no register
%allocation or instruction scheduling is considered. Obtaining registers
%for new instructions is achieved by pushing register values to the stack, after
%the execution of a new instruction the registers previous values are retrieved
%from the stack.



\begin{itemize}
    \item naive implementation description \\
The naive transformation is very simplistic and does not consider performance
in any form, its only goal is to produce an executable code.
    \item ---------------------------------------
    \item region transform \\
The transformation is applied on a group or region of inserted instructions.
For the group of inserted instructions the transformer will exchange all
symbolic registers present.
    \item register allocation \\
The register allocation for a group is simple. Each symbolic register encountered
is replaced with a physical register throughout the region. To preserve the original
data flow and execution the values of the physical registers now used in the
instruction group needs to be save before using them and restored afterwards.
This is done by inserting load and store instructions at the beginning and end
of the region to preserve the register values. The content of these registers
are saved to a space on the stack which the naive transformer allocates.
    \item special case, using accumulator register \\
A special case regarding the allocation of register is when a region contains
instructions that uses the 64-bit accumulator register. If this is used
then a regular register needs to be used to move values to and from
the high and low portion of the accumulator register. This register that
needs to be used can be one already allocated for instructions in the
region. However if the no regular register will be used in the region then
an extra register needs to be used just to be able to preserve the accumulator
register.
    \item stack modification \\
To be able to save register values space on the stack is needed. To know
how much space is needed the maximum number of symbolic registers
encountered in a region is used. This will ensure that there will always
be enough space on the stack to save register values for each region.
When it is known how much stack space is needed the instructions
responsible for stack allocation and deallocation are adjusted to
include the new stack space used by the naivetransformer.
    \item ---------------------------------------
    \item need to remember limitations i have done. how do i write it?
just add it to the text?
\end{itemize}




\section{Optimized Transformation}
%what is the optimized tranformation using.
To improve the execution of transformed binaries the optimized transformation
can be used. Compared to the naive transformation the optimized uses optimizations
implemented in the frameworkm, such as register allocation and instruction scheduling
to improve the execution time by. With the optimizations the utilization of register
can be improved and stalls in the MIPS pipeline can be avoided.

\todo[inline]{These subsections might not be needed, include in above text instead}

\subsection{Register Allocation}
%What register allocation is implemented?

\subsection{Instruction Scheduling}
%What instruction scheduling is implemented?

