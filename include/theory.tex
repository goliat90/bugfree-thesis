\chapter{Theory}

\todo[inline]{Change this chapter to a pure theory chapter that handles
theoretical materials needed to understand the project}
\begin{itemize}
    \item programs, basic blocks.
    %
    \item compiler optimizations
    \item Describe different kinds of optimizations
    %
    \item -----------------------
    %
    \item three address code
    \item static single assignment
    \item ssa:phi functions
    \item register allocation
    \item graph coloring 
    %
    \item -----------------------
    %
    \item instruction scheduling
    \item direct acyclic graphs, or similar, components used
    \item global and block scheduling
    %
    \item -----------------------
    %
    \item postpass \& prepass
    %
    \item -----------------------
    %
    
    %
    \item fault tolerance \\
\end{itemize}

Register allocation order.
\begin{itemize}
    \item ----------------------------------------
    \item Register allocation intro
    \item Common terminologies
    \item Graph coloring
        \item Spilling
        \item Coalesing
    \item Aliasing
    \item Live ranges
    \item ----------------------------------------
\end{itemize}

\todo[inline]{Write introduction for chapter, theory.}


\section{Compiler Optimizations}
\todo[inline]{Introduction to optimizations, their purpose.}
During compilation of a program optimizations are applied
to improve the execution time the binary. Two known categories
of optimization is register allocation and instruction scheduling.
To improve the execution of transformed binaries optimizations
like these could also be applied.

\subsection{Register Allocation}
\todo[inline]{Write about the choices of register allocation}


\subsection{Instruction Scheduling}
\todo[inline]{Write about the choices of instruction scheduling}


\section{MIPS}
\todo[inline]{Introduction to mips. This could be unneccesary if
considering the level of readers knowledge}

\section{ROSE}
\todo[inline]{Introduction to rose, its capabilities and what is
it is used for}
ROSE is an open-source compiler infrastructure which can be used to build
source-to-source compilers for a large set of programming languages.
It is also capable of performing binary analysis on several
binary formats, one of them being MIPS.
%

%By using ROSE as the foundation of the framework
\todo[inline]{Maybe remove this paragrah and move it to implementation}
Its frontend is used in the project for parsing the binaries and using
the abstract syntax tree(AST) that ROSE creates to apply desired
transformations to the binary. The backend will also be used to build
binaries after transformations have been applied~\cite{rose}.

\section{gem5}
\todo[inline]{introduction to gem5, what it offers etc.}
The gem5 is the simulator used during the project to evaluate the performance
of transformed binaries. It was selected since it supports MIPS and
is also acknowledged in academic community as well as industry~\cite{gem5}.

\section{Fault-Tolerance}
\todo[inline]{short description of the tmr fault tolerance used,
have a reference as well?}

